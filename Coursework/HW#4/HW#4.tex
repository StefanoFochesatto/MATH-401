%%%%%%%%%%%%%%%%%%%%%%%%%%%%%%%%%%%%%%%%%%%%%%%%%%%%%%%%%%%%%%%%%%%%%%%%%%%%%%%%%%%%%%%
%%%%%%%%%%%%%%%%%%%%%%%%%%%%%%%%%%%%%%%%%%%%%%%%%%%%%%%%%%%%%%%%%%%%%%%%%%%%%%%%%%%%%%%
% 
% This top part of the document is called the 'preamble'.  Modify it with caution!
%
% The real document starts below where it says 'The main document starts here'.

\documentclass[12pt]{article}

\usepackage{amssymb,amsmath,amsthm}
\usepackage[top=1in, bottom=1in, left=1.25in, right=1.25in]{geometry}
\usepackage{fancyhdr}
\usepackage{enumerate}

% Comment the following line to use TeX's default font of Computer Modern.
\usepackage{times,txfonts}

\newtheoremstyle{homework}% name of the style to be used
  {18pt}% measure of space to leave above the theorem. E.g.: 3pt
  {12pt}% measure of space to leave below the theorem. E.g.: 3pt
  {}% name of font to use in the body of the theorem
  {}% measure of space to indent
  {\bfseries}% name of head font
  {:}% punctuation between head and body
  {2ex}% space after theorem head; " " = normal interword space
  {}% Manually specify head
\theoremstyle{homework} 

% Set up an Exercise environment and a Solution label.
\newtheorem*{exercisecore}{Exercise \@currentlabel}
\newenvironment{exercise}[1]
{\def\@currentlabel{#1}\exercisecore}
{\endexercisecore}

\newcommand{\localhead}[1]{\par\smallskip\noindent\textbf{#1}\nobreak\\}%
\newcommand\solution{\localhead{Solution:}}

%%%%%%%%%%%%%%%%%%%%%%%%%%%%%%%%%%%%%%%%%%%%%%%%%%%%%%%%%%%%%%%%%%%%%%%%
%
% Stuff for getting the name/document date/title across the header
\makeatletter
\RequirePackage{fancyhdr}
\pagestyle{fancy}
\fancyfoot[C]{\ifnum \value{page} > 1\relax\thepage\fi}
\fancyhead[L]{\ifx\@doclabel\@empty\else\@doclabel\fi}
\fancyhead[C]{\ifx\@docdate\@empty\else\@docdate\fi}
\fancyhead[R]{\ifx\@docauthor\@empty\else\@docauthor\fi}
\headheight 15pt

\def\doclabel#1{\gdef\@doclabel{#1}}
\doclabel{Use {\tt\textbackslash doclabel\{MY LABEL\}}.}
\def\docdate#1{\gdef\@docdate{#1}}
\docdate{Use {\tt\textbackslash docdate\{MY DATE\}}.}
\def\docauthor#1{\gdef\@docauthor{#1}}
\docauthor{Use {\tt\textbackslash docauthor\{MY NAME\}}.}
\makeatother

% Shortcuts for blackboard bold number sets (reals, integers, etc.)
\newcommand{\Reals}{\ensuremath{\mathbb R}}
\newcommand{\Nats}{\ensuremath{\mathbb N}}
\newcommand{\Ints}{\ensuremath{\mathbb Z}}
\newcommand{\Rats}{\ensuremath{\mathbb Q}}
\newcommand{\Cplx}{\ensuremath{\mathbb C}}
%% Some equivalents that some people may prefer.
\let\RR\Reals
\let\NN\Nats
\let\II\Ints
\let\CC\Cplx

%%%%%%%%%%%%%%%%%%%%%%%%%%%%%%%%%%%%%%%%%%%%%%%%%%%%%%%%%%%ss%%%%%%%%%%%%%%%%%%%%%%%%%%%
%%%%%%%%%%%%%%%%%%%%%%%%%%%%%%%%%%%%%%%%%%%%%%%%%%%%%%%%%%%%%%%%%%%%%%%%%%%%%%%%%%%%%%%
% 
% The main document start here.

% The following commands set up the material that appears in the header.
\doclabel{Math 401: Homework 4}
\docauthor{Stefano Fochesatto}
\docdate{September 21, 2020}

\begin{document}

\begin{exercise}{Supplemental 1} 
Show that the sequence $(-1)^n$ does not converge.
\end{exercise}
\begin{proof} Suppose for the sake of contradiction that the sequence $(-1)^n$ converges to $l$. By the definition of 
	converges we know that for all tolerances $\epsilon \in \Reals$ there exists some $N\in \Nats$ such that for all $n \leq N$,
	\begin{equation*}
		|L - (-1)^n| < \epsilon.
	\end{equation*}
	Consider $\epsilon  = 2$ and suppose $(-1)^n = 1$, then,
	\begin{align*}
		|L - 1| &< \dfrac{1}{2},\\
		-\dfrac{1}{2} < &L - 1 < \dfrac{1}{2},\\
		\dfrac{1}{2} < &L < \dfrac{3}{2}.
	\end{align*}
	Now suppose that $(-1)^n = -1$
	\begin{align*}
		|L + 1| &< \dfrac{1}{2},\\
		-\dfrac{1}{2}< &L + 1 < \dfrac{1}{2},\\
		-1 < &L < -\dfrac{1}{2}.
	\end{align*} 
Clearly $L$ cannot exist in both $(-1,-\frac{1}{2})$ and $(\frac{1}{2}, \frac{3}{2})$ thus a contradiction. 
\end{proof}

\begin{exercise}{Supplemental 2} \strut
\begin{enumerate}[(a)]
	\item Show that for all $n\in\Nats$, $2^n\ge n$.
	\item Show  that $\lim_{n\to\infty} 1/2^n = 0$.
\end{enumerate}
\end{exercise}
\begin{proof}[Part (a)]
	Consider the case where $n = 1$. Clearly,
	\begin{align*}
		2^{(1)} &= 2\\ 
		&\geq (1).
	\end{align*}
	Now we will proceed by induction on $n$. Suppose there exists some $n \in \Nats$ such that,
	\begin{equation*}
		2^n \geq n.
	\end{equation*}
	Now note that,
	\begin{align*}
		2^n &\geq n, \\
		2^n + 1 &\geq n + 1,  \\
		2^n + 2^n &\geq n + 1,\\
		2^n2 &\geq n + 1,\\
		2^{n+1} &\geq n + 1.
	\end{align*} 
	Thus by induction we have shown that for all $n \in \Nats $ $2^n \geq n$. 
\end{proof}
\begin{proof}[Part (b)] Let $\epsilon > 0$. Pick $N \in \Nats$ such that $\frac{1}{2^N} < \epsilon$. Then for all $n \geq N$,
	\begin{align*}
		|0 - \frac{1}{2^n}| &= \frac{1}{2^n},\\
							&\le \frac{1}{2^N},\\
							&< \epsilon.
	\end{align*}
	Thus the sequence $\frac{1}{2^n}$ converges to 0.  
\end{proof}












\begin{exercise}{2.2.2}
From the definition, compute the given limits.
\begin{enumerate}
	\item[\textbf{a.}] 
	\begin{equation*}
		\lim \dfrac{2n+1}{5n+4} = \dfrac{2}{5}
	\end{equation*} 
	\item[\textbf{b.}]
	\begin{equation*}
		\lim \dfrac{2n^2}{n^3+3} = 0 
	\end{equation*}
	\item[\textbf{c.}] 
	\begin{equation*}
		\lim \dfrac{sin(n^2)}{n^\frac{1}{3}} = 0 
	\end{equation*} 
\end{enumerate}
\end{exercise}
\begin{proof}[Part (a)] let $\epsilon > 0$. Note that,
	\begin{equation*}
		|\dfrac{2}{5} - \dfrac{2n+1}{5n+4}| = \dfrac{3}{5(5n+4)} < \dfrac{3}{5n}.
	\end{equation*}
	Through Theorem 1.4.2(ii) we can pick $N \in \Nats$ such that $\dfrac{1}{N}<\dfrac{5\epsilon}{3}$. Then for all $n \geq N$,
	\begin{align*}
		|\dfrac{2}{5} - \dfrac{2n+1}{5n+4}| &= \dfrac{3}{5(5n+4)},\\
		&<\dfrac{3}{5n},\\
		&<\dfrac{3}{5N},\\
		&< \epsilon.
	\end{align*}
	Thus the sequence $\frac{2n+1}{5n+4}$ converges to $\frac{2}{5}$.  
\end{proof}
\begin{proof}[Part (b)] let $\epsilon > 0$. Note that,
	\begin{equation*}
		|0 - \dfrac{2n^2}{n^3+3}| = \dfrac{2n^2}{n^3+3} \le \dfrac{2n^2}{n^3} = \dfrac{2}{n}.
	\end{equation*}
	Through Theorem 1.4.2(ii) we can pick $N \in \Nats$ such that $\dfrac{2}{N}<\epsilon$. Then for all $n \geq N$,
	\begin{align*}
		|0 - \dfrac{2n^2}{n^3+3}| &= \dfrac{2n^2}{n^3+3},\\
		&\le \dfrac{2}{N},\\
		&< \epsilon.
	\end{align*}
	Thus the sequence $\dfrac{2n^2}{n^3+3}$ converges to $0$.  
\end{proof} 
\begin{proof}[Part (c)]let $\epsilon > 0$. Note that the inequality,
	\begin{equation*}
		|0 - \dfrac{sin(n^2)}{n^\frac{1}{3}}| = \dfrac{sin(n^2)}{n^\frac{1}{3}} \le \dfrac{1}{n^\frac{1}{3}}. 
	\end{equation*}
	Through Theorem 1.4.2(ii) we can pick $N \in \Nats$ such that $\dfrac{1}{N}<\epsilon^3$. Then for all $n \geq N$,
	\begin{align*}
		|0 - \dfrac{sin(n^2)}{n^\frac{1}{3}}| &= \dfrac{sin(n^2)}{n^\frac{1}{3}},\\
		&\le \dfrac{1}{n^\frac{1}{3}},\\
		&\le \dfrac{1}{N^\frac{1}{3}},\\
		&< \epsilon.
	\end{align*}
	Thus the sequence $\dfrac{sin(n^2)}{n^\frac{1}{3}}$ converges to $0$.  
\end{proof}











\begin{exercise}{2.2.3}
Describe what needs to be shown to disprove the given statements.
\end{exercise}
\solution
\begin{enumerate}[(a)]
\item Find a college in the United States where every student is less than 7 feet tall.
\item Find a college in the United States where no professor gives their students an $A$ or $B$.
\item show that for all colleges in the United States, there exists some student who is less than 6 feet tall.
\end{enumerate}

\begin{exercise}{2.2.6}
Prove Theorem 2.2.7. To get started, assume $(a_n) \to a$ and also that $(a_n) \to b$ and prove that $a = b$
\end{exercise}
\begin{proof} Suppose $(a_n)$ is a convergent series where $(a_n) \to a$ and also that $(a_n) \to b$. By the definition of convergence we know that
	there exist some $\epsilon > 0$ where for $N_a\in \Nats$ , and that for all $n \geq N_a$ then, 
	\begin{equation*}
		|a - a_n| < \frac{\epsilon}{2}
	\end{equation*}
	Likewise there exists some $N_b\in \Nats$, where for all $n \geq N_b$ such that,
	\begin{equation*}
		|b - a_n| < \frac{\epsilon}{2}
	\end{equation*} 
	If $N = max\{N_a,N_b\}$ then for all $n \geq N$ we know that both inequalities hold. Now through some algebra and the triangle inequality we get,
	\begin{align*}
		|a - b| &= |a - a_n + a_n - b|\\
		&\le |a - a_n| + |a_n - b|\\
		&< \epsilon.
	\end{align*}
Note that we have shown that, 
\begin{equation*}
	|a - b| < \epsilon
\end{equation*}
is true for all $\epsilon > 0$ and thus as a consequence it must be the case that,
\begin{align*}
	|a - b| &= 0,\\
	a &= b.
\end{align*}
\end{proof}

\begin{exercise}{2.2.5(a)}
Determine, with a proof, $\lim_{n\to\infty} [[5/n]]$.
\end{exercise}
\solution
Claim: From calculating the first few numbers in the sequence I get,
\begin{equation*}
	5,2,1,1,1,0,0
\end{equation*}
Therefore I claim that $\lim_{n\to\infty} [[5/n]] = 0$

\begin{proof} Let $\epsilon > 0$. Note that as long as we go out more than $5$ elements in the sequence then the convergence condition is satisfied. Let $N = 6$ and note that for all $n\geq N$,
	\begin{align*}
		|0 - [[5/n]]| &= [[5/n]],\\
					&= [[5/N]],\\
					&= 0,\\
					&< \epsilon.
	\end{align*} 
Thus the sequence $[[5/n]]$ converges to 0.
\end{proof}


\begin{exercise}{2.3.9(a)(c)}
\strut
\begin{enumerate}
	\item[(a)] If $(a_n)$ is a bounded sequence and $b_n\to 0$,
	show $a_nb_n\to 0$.
	\item[(c)] Prove Theorem 2.3.3(iii) for the case $a=0$.
\end{enumerate}
\end{exercise}
\solution
\begin{enumerate}
	\item[(a)] 
	\begin{proof} Suppose that $(a_n)$ is a bounded sequence and $b_n\to 0$. Since $(a_n)$ is bounded, there exists some $M \in \Reals$
		such that $a_n \le M$ for all $n$. Since $b_n \to 0$ we know that for all $\epsilon > 0$ there exists some $N \in \Nats$ such that for all $n \geq N$, 
		\begin{equation*}
			|0 - b_n| = |b_n| < \frac{\epsilon}{M}.
		\end{equation*} 
		Through some algebra we can see that, 
		\begin{align*}
			|a_nb_n| &= |a_n||b_n|,\\
			&\le M|b_n|,\\
			&< M\dfrac{\epsilon}{M},\\ 
			&= \epsilon.  
		\end{align*}
		Note that we have shown that, $|a_nb_n| < \epsilon$ thus $a_nb_n\to 0$.


	\end{proof}
	\item[(c)]
	\begin{proof} Suppose a sequence $(a_n)$ and $(b_n)$ such that $a_n\to 0$ and $b_n\to b$. Note that since $b_n$ converges it must be bounded and therefore from the prevous proof we get
		that,
		\begin{equation*}
	\lim a_nb_n = 0 = 0b.
		\end{equation*}
		
	\end{proof}
\end{enumerate}


\end{document}
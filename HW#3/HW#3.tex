
%%%%%%%%%%%%%%%%%%%%%%%%%%%%%%%%%%%%%%%%%%%%%%%%%%%%%%%%%%%%%%%%%%%%%%%%%%%%%%%%%%%%%%%
%%%%%%%%%%%%%%%%%%%%%%%%%%%%%%%%%%%%%%%%%%%%%%%%%%%%%%%%%%%%%%%%%%%%%%%%%%%%%%%%%%%%%%%
% 
% This top part of the document is called the 'preamble'.  Modify it with caution!
%
% The real document starts below where it says 'The main document starts here'.

\documentclass[12pt]{article}

\usepackage{amssymb,amsmath,amsthm}
\usepackage[top=1in, bottom=1in, left=1.25in, right=1.25in]{geometry}
\usepackage{fancyhdr}
\usepackage{enumerate}

% Comment the following line to use TeX's default font of Computer Modern.
\usepackage{times,txfonts}

\newtheoremstyle{homework}% name of the style to be used
  {18pt}% measure of space to leave above the theorem. E.g.: 3pt
  {12pt}% measure of space to leave below the theorem. E.g.: 3pt
  {}% name of font to use in the body of the theorem
  {}% measure of space to indent
  {\bfseries}% name of head font
  {:}% punctuation between head and body
  {2ex}% space after theorem head; " " = normal interword space
  {}% Manually specify head
\theoremstyle{homework} 

% Set up an Exercise environment and a Solution label.
\newtheorem*{exercisecore}{Exercise \@currentlabel}
\newenvironment{exercise}[1]
{\def\@currentlabel{#1}\exercisecore}
{\endexercisecore}

\newcommand{\localhead}[1]{\par\smallskip\noindent\textbf{#1}\nobreak\\}%
\newcommand\solution{\localhead{Solution:}}

%%%%%%%%%%%%%%%%%%%%%%%%%%%%%%%%%%%%%%%%%%%%%%%%%%%%%%%%%%%%%%%%%%%%%%%%
%
% Stuff for getting the name/document date/title across the header
\makeatletter
\RequirePackage{fancyhdr}
\pagestyle{fancy}
\fancyfoot[C]{\ifnum \value{page} > 1\relax\thepage\fi}
\fancyhead[L]{\ifx\@doclabel\@empty\else\@doclabel\fi}
\fancyhead[C]{\ifx\@docdate\@empty\else\@docdate\fi}
\fancyhead[R]{\ifx\@docauthor\@empty\else\@docauthor\fi}
\headheight 15pt

\def\doclabel#1{\gdef\@doclabel{#1}}
\doclabel{Use {\tt\textbackslash doclabel\{MY LABEL\}}.}
\def\docdate#1{\gdef\@docdate{#1}}
\docdate{Use {\tt\textbackslash docdate\{MY DATE\}}.}
\def\docauthor#1{\gdef\@docauthor{#1}}
\docauthor{Use {\tt\textbackslash docauthor\{MY NAME\}}.}
\makeatother

% Shortcuts for blackboard bold number sets (reals, integers, etc.)
\newcommand{\Reals}{\ensuremath{\mathbb R}}
\newcommand{\Nats}{\ensuremath{\mathbb N}}
\newcommand{\Ints}{\ensuremath{\mathbb Z}}
\newcommand{\Rats}{\ensuremath{\mathbb Q}}
\newcommand{\Cplx}{\ensuremath{\mathbb C}}
%% Some equivalents that some people may prefer.
\let\RR\Reals
\let\NN\Nats
\let\II\Ints
\let\CC\Cplx

%%%%%%%%%%%%%%%%%%%%%%%%%%%%%%%%%%%%%%%%%%%%%%%%%%%%%%%%%%%%%%%%%%%%%%%%%%%%%%%%%%%%%%%
%%%%%%%%%%%%%%%%%%%%%%%%%%%%%%%%%%%%%%%%%%%%%%%%%%%%%%%%%%%%%%%%%%%%%%%%%%%%%%%%%%%%%%%
% 
% The main document start here.

% The following commands set up the material that appears in the header.
\doclabel{Math 401: Homework 3}
\docauthor{Stefano Fochesatto}
\docdate{\today}

\begin{document}


\begin{exercise}{1.4.7} Finish the proof of Theorem 1.4.5 by
showing that the assumption $\alpha^2>2$ contradicts the
assumption that $\alpha=\sup A$.
\end{exercise}
\begin{proof} Consider the set,
  \begin{equation*}
    A = \{a \in \Reals : a^2<2\}.
  \end{equation*}
Let $\alpha = \sup A$. Suppose to the contrary that $\alpha^2>2$. Consider an element 
of $A$ that is smaller than $\alpha$, like $(\alpha - \frac{1}{n})$, where $n > \frac{2\alpha}{\alpha^2 - 2}$.
\begin{align*}
  (\alpha - \frac{1}{n})^2 &= \alpha^2 - \frac{2\alpha}{n} + \frac{1}{n^2},\\
  &> \alpha^2 - \frac{2\alpha}{n},\\
  &> \alpha^2 - (\alpha^2 - 2),\\
  &= 2.
\end{align*}
Thus we have shown that $ (\alpha - \frac{1}{n})$ is greater than $a$ for all $a \in A$ and therefore $ (\alpha - \frac{1}{n})$ is an upper bound.
Since $(\alpha - \frac{1}{n})<\alpha$ we have contradicted $\alpha = \sup A$.


\end{proof}

\begin{exercise}{Supplemental 1} Give a from-scratch
proof of the following facts:
\begin{enumerate}[(a)]
\item If $f:A\to B$ has an inverse function $g$, then
$f$ is injective.
\item If $f:A\to B$ has an inverse function $g$, then
$f$ is surjective.
\end{enumerate}
\end{exercise}
\begin{proof}[Proof (a)]
Suppose $f:A\to B$, whose inverse is $g:B\to A$, now consider $a_i,a_j \in A$ such that $f(a_i) = f(a_j)$. Using $g$ as 
an intermediary we get the equality,
\begin{align*}
  f(a_i) &= f(a_j),\\
  g(f(a_i)) &= g(f(a_j)),\\
  a_i &= a_j.
\end{align*}
Thus we have shown $f$ is an injective function.
\end{proof}
\begin{proof}[Proof (b)]
  Suppose $f:A \to B$, whose inverse is $g:B \to A$. Consider some $b \in B$, by definition of $g$ we know that
  there exists some $a \in A$ such that, $g(b) = a$. Taking the inverse of both sides we get,
  \begin{align*}
    f(g(b)) &= f(a)\\
    b &= f(a).
  \end{align*}
  Since $a \in A$ we have shown that for every $b \in B$ there exists some $a \in A$ such that $f(a) = b$ thus $f$ is surjective. 
\end{proof}

\begin{exercise}{Supplemental 2} Show that the sets $[0,1)$ and $(0,1)$ have the same cardinality.\\
  \begin{proof}
    Suppose the function $f:[0,1) \to (0,1) $ defined by,
  \[ f(x) = \begin{cases} 
    1 - \frac{1}{n+1}  & 1 - \frac{1}{n},  n \in \Nats\\
    x & \text{Otherwise} 
 \end{cases}
\]
Suppose $a,b \in [0,1)$ such that $f(a) = f(b)$. For the case where $f(x) = x$ the function is trivially injective. let $a,b$ be of the form $1 - \frac{1}{n}$ such that,
$a = 1 - \frac{1}{n}$ and $b = 1 - \frac{1}{m}$, where $n,m \in \Nats$. Now consider $f(a) = f(b)$, by definition,
\begin{align*}
  f(a) &= f(b),\\
  1 - \frac{1}{n+1} &=  1 - \frac{1}{m+1},\\
  \frac{1}{n+1} &= \frac{1}{m+1},\\
  n+1 &= m+1,\\
  n &= m.
\end{align*} 
Thus we have shown that $n = m$ and therefore by substitution $a = b$. Thus $f$ is injective.
\vspace{.25in}



Note that for the case where $f(x) = x$ the function is trivially surjective. Suppose some $b \in (0,1)$. Let $b$ be of the form $b = 1 - \frac{1}{n+1}$, where $n \in \Nats$. Consider $a = 1 - \frac{1}{n}$ and note that, $f(a) = b$. Observe that $a \in [0,1)$,
thus $f$ is surjective.
  \end{proof}
\end{exercise}


\begin{exercise}{1.5.10 (a) (c)}
(Wait until after Wednesday to start this one)
\begin{enumerate}[(a)] 
	
\item Let $C\subseteq[0,1]$ be uncountable.  Show that
there exists $a\in(0,1)$ such that $C\cap [a,1]$ is uncountable.
\item[(c)] Determine, with proof, if the same statement remains
true replacing uncountable with infinite.
\end{enumerate}
\end{exercise}
\begin{proof}[Proof (a)] Suppose $C \subseteq [0,1]$ is uncountable. Suppose for the sake of contradiction that for all $a \in (0,1)$,
  $C \cap [a,1]$ is countable. Let $a = \frac{1}{n}$. Note that $C \cap [\frac{1}{n},1]$ is countable.  By Theorem 1.5.8 we know that since $C \cap [\frac{1}{n},1]$ is countable the infinite union is also countable. Through set theory
  \begin{align*}
    \bigcup_{n - 1}^{\infty} C \cap [\frac{1}{n},1] &= C \cap (\bigcup_{n - 1}^{\infty} [\frac{1}{n},1])\\
    &=  C \cap (0,1]
  \end{align*}
Therefore, $C \cap (0,1] \cup \{0\} = C$ is also countable. 
\end{proof}
\begin{proof}[Proof (c)] Suppose the countably infinite set $C = \{\frac{1}{n}, n \in \Nats\}$. By Archimedean Principle we know that 
  for all $a \in (0,1)$ we can find some $\frac{1}{n}<a$, and therefore we force the set $C \cap [a, 1]$ to be finite. 




\end{proof}

\begin{exercise}{Supplemental 3}
(Wait until after Wednesday to start this one)
Suppose for each $k\in\Nats$ that $A_k$ is at most countable.
Use the fact that $\Nats\times\Nats$ is countably infinite to
show that $\cup_{k=1}^\infty A_k$ is at most countable.  Hint:
take advantage of surjection.
\end{exercise}
\begin{proof}
  Suppose for each $k\in\Nats$ that $A_k$ is at most countable. Recall that since all $A_k$ are at most countable there must exists a surjection $g_k: \Nats \to A_k$.  Consider the function $f: \Nats\times\Nats \to \cup_{k=1}^\infty A_k$ defined such that,
  $f(n,m) = g_n(m)$. Let $a \in \cup_{k=1}^\infty A_k$ and by definition we know that $a$ must exist in some set $A_i$, where $i \in \Nats$. Furthermore, since $g_i$, is a surjection we know that there exists some $j \in \Nats$ where $g_i(j) = a$. Therefore we know that $f(i,j) = g_i(j)$ where $i,j \in \Nats\times\Nats$. Thus $f$ is a surjection, and it follows that since
   $\Nats\times\Nats$ is countably infinite then $ \cup_{k=1}^\infty A_k$ is at most countable. 
\end{proof}

\end{document}